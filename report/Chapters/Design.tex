% !TeX spellcheck = en_GB
\chapter{Design}
\lhead{\emph{Design}}

\section{Algorithm Design}
As the project was originally intended for Haskell, it was decided that a recursive implementation of LSBF would be written for ease of porting in the future.

Nested bit arrays were used as a space efficient way to store the set bits. From a high-level perspective, bits are set based on the output of the hash of the input, $hash_i(input)$. The number of hashes is important in determining run-time and likelihood of false positive. If a minimum false positive rate is desired while optimising the number of hashes, then $k$ depends on $m$ and the expected $n$ as given below.\cite{ProbabilityAndComputing}
\begin{center}
	$ k = \dfrac{m}{n}\ln 2 $
\end{center}
We can calculate the optimum number of hashes to use based on our tolerated probability of a false positive rate $p$ using the following formulae:\cite{1195150}

\begin{center}
	$p=\left( 1-e^{-(\frac{m}{n}\ln 2)\frac{n}{m}} \right)^{\frac{m}{n}\ln 2} $
\end{center}

which simplifies to

\begin{center}
$\ln p = -\frac{m}{n}(\ln 2)^2$
\end{center}

meaning that

\begin{center}
$m = -\dfrac{n\ln p}{(\ln 2)^2}$
\end{center}

Finally, $k$ for our desired $p$ can be derived as

\begin{center}
	$k = -\dfrac{\ln p}{\ln 2}$
\end{center}

Here, we can see that for a given $p$, $k$ only depends on $p$, and $m \propto n$.

For ease of implementation two common hashes were used to prove how LSBF would work, however suggestions on more suitable hashes will follow in Chapter \ref{Evaluation}: Evaluation. Additionally the above formulae work for regular Bloom filters but would need to be adapted for LSBF. However, for establishing a rough baseline, this would do fine.

\section{Pseudocode}

While Mitzenmacher's Distance-Sensitive Bloom Filter has both a false positive and false negative rate\cite{DSBF}, this version of LSBF makes an effort to eliminate false negative by use of rounding. In the pseudocode notation, we will use $\alpha$ to denote the upper bound for locality and $\beta$ for the rounding distance.

\begin{algorithm}[H]
	\KwIn{$ k, m , \alpha, \beta $}
	\KwResult{ An LSBF Object}
	\If{$k > 0$ and $m > 0$}{
		initialize an array of $m$ zeros\;
		store $\alpha, \beta, k$ in the object\;
		\KwRet{a reference to new LSBF object}
	}
	\caption{LSBF($k, m, \alpha, \beta$), LSBF Set Up}
\end{algorithm}

Adding an element to the LSBF is trivial in abstract. We assume that a function ``round'' exists that rounds to the nearest $\beta$, the second parameter:

\begin{algorithm}[H]
	\KwIn{input}
	\KwResult{None}	
	$input$ := $round(input, \beta)$\;
	\ForEach{hash, up to $k$}{
		Set a bit in the object's array to 1 based on the result of \\
		$\;\;hash(input)\mod{m}$\;
	}
\caption{add($ input $),Adding an element to the LSBF}
\end{algorithm}

In order to make the next section of code easier to understand, we will state a function, $checkInput$, that hashes for each $k$ and returns True when all checked bits are set to one.

\begin{algorithm}
	\KwIn{input}
	\KwResult{True when the bits defined by $hash_i(input)$ are all one for every $i$, False otherwise}
	$input$ := $round(input, \beta)$\;
	$i$ := 0\;
	\While{$i < k$}{
		\eIf{the bit given by $hash_i$(input) is one}{
			$i$ := $i$ + 1\;
		}{
			\KwRet{False}
		}
	}
	\KwRet{True}
	\caption{checkInput($ input $), Checking bits when testing for presence}
\end{algorithm}

Checking if a value is in the LSBF is, in this implementation, a recursive check of the space from $input - \alpha$ to $input + \alpha$ with a reduction of $\alpha$ by $\beta$ in each successive recursion until $\alpha$ is less than $\beta$.

\begin{algorithm}[H]
	\KwIn{input, $\alpha$}
	\KwResult{True when an iteration shows all checked bits set to one, False otherwise}
	$input$ := $round(input, \beta)$
	
	\eIf{$\alpha < \beta$}{
		\KwRet{$checkInput(input)$}
	}{
		\If{not $checkInput(input-\beta)$}{
			\If{not $checkInput(input + \beta)$}{
				\If{not $recursiveInputCheck(input, \alpha - \beta)$}
				{\KwRet{False}}
			}
		}
		\KwRet{True}
	}
\caption{recursiveInputCheck(input, $\alpha$)}
\end{algorithm}

\section{Example}

To illustrate this algorithm in action, we will run through an example here. This will follow a Python-like syntax, but will not be actual Python code.

\begin{lstlisting}[language=bash]
  #Creates an LSBF object that uses 2 hashes, has a bit space of 10000,
  # and will check whole numbers in a range plus-minus 5 the query value
  > myLSBF = LSBF(2, 10000, 5, 1)
  
  #Add values to the filter
  > myLSBF.add(12) #now up to 2 bits are set
  > myLSBF.add(42) #up to 4 bits are set
  
  #We may use this as a normal Bloom Filter
  > myLSBF.checkInput(12)
  < True
  > myLSBF.checkInput(14)
  < False
  
  #But it also rounds to the nearest whole number
  > myLSBF.checkInput(22.1)
  < Checking 12: True
  
  #Now the recursive locality check with some verbose output
  #An actual implementation would use the already defined alpha, as in Appendix A
  > myLSBF.recursiveInputCheck(12, 5)
  < Alpha: 5; Beta:1;
  < Checking (12 - 5): False
  < Checking (12 - 5): False
  < Alpha: 4; Beta:1;
  < Checking (12 - 4): False
  < Checking (12 - 4): False
  < Alpha: 3; Beta:1;
  < Checking (12 - 3): False
  < Checking (12 - 3): False
  < Alpha: 2; Beta:1;
  < Checking (12 - 2): False
  < Checking (12 - 2): False
  < Alpha: 1; Beta:1;
  < Checking (12 - 1): False
  < Checking (12 - 1): False
  < Alpha: 0; Beta:1;
  < Alpha < Beta; Checking 12: True
  
  
\end{lstlisting}
