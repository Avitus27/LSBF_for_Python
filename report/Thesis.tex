% !TeX spellcheck = en_GB
% Set up the document
\documentclass[a4paper, 12pt, oneside]{Thesis}  % Use the "Thesis" style, based on the ECS Thesis style by Steve Gunn
\graphicspath{Figures/}  % Location of the graphics files (set up for graphics to be in PDF format)
\newcommand{\Mod}[1]{\ (\mathrm{mod}\ #1)}
% Include any extra LaTeX packages required
\usepackage[square, numbers, comma, sort&compress]{natbib }  % Use the "Natbib" style for the references in the Bibliography
\usepackage[lined]{algorithm2e}
\usepackage{verbatim}  % Needed for the "comment" environment to make LaTeX comments
\usepackage{vector}  % Allows "\bvec{}" and "\buvec{}" for "blackboard" style bold vectors in maths
\hypersetup{urlcolor=blue, colorlinks=true}  % Colours hyperlinks in blue, but this can be distracting if there are many links.

%% ----------------------------------------------------------------
\begin{document}
\frontmatter      % Begin Roman style (i, ii, iii, iv...) page numbering

% Set up the Title Page
\title  {An Implementation of Locality Sensitive Bloom Filter}
\authors  {\texorpdfstring
            {\href{mailto:113315141@umail.ucc.ie}{Mervyn Edmond Galvin}}
            {Mervyn Edmond Galvin}
            }
\addresses  {\groupname\\\deptname\\\univname}  % Do not change this here, instead these must be set in the "Thesis.cls" file, please look through it instead
\date       {\today}
\subject    {}
\keywords   {}

\maketitle
%% ----------------------------------------------------------------

\setstretch{1.3}  % It is better to have smaller font and larger line spacing than the other way round

% Define the page headers using the FancyHdr package and set up for one-sided printing
\fancyhead{}  % Clears all page headers and footers
\rhead{\thepage}  % Sets the right side header to show the page number
\lhead{}  % Clears the left side page header

\pagestyle{fancy}  % Finally, use the "fancy" page style to implement the FancyHdr headers

%% ----------------------------------------------------------------

% The Abstract Page
\addtotoc{Abstract}  % Add the "Abstract" page entry to the Contents
\abstract{
	\addtocontents{toc}{\vspace{1em}}  % Add a gap in the Contents, for aesthetics
	A bloom filter is a space efficient probabilistic data structure which can quickly test for set membership.
	
	However, since a standard bloom filter will only test for presence, it has limitations. This project shows the feasibility of Locality Sensitive Bloom Filters (LSBF) for single-dimensional data and points to further, theoretical, algorithms for distance sensitive hashes that can be used efficiently for multidimensional data. This report also contains an approximate calculation of space and time complexity.
	
	This has applications in areas like machine learning and searching where an answer that’s “close enough” may be satisfactory.
}

\clearpage  % Abstract ended, start a new page
%% ----------------------------------------------------------------

% Declaration Page required for the Thesis, your institution may give you a different text to place here
\Declaration{

\addtocontents{toc}{\vspace{1em}}  % Add a gap in the Contents, for aesthetics
In signing this declaration, you are conforming, in writing, that the submitted work is entirely your own original work, except where clearly attributed otherwise, and that it has not been submitted partly or wholly for any other educational award.

I hereby declare that:
\begin{itemize} 
\item[\tiny{$\blacksquare$}] this is all my own work, unless clearly indicated otherwise, with full and proper accreditation;
 
\item[\tiny{$\blacksquare$}] with respect to my own work: none of it has been submitted at any educational institution contributing in any way to an educational award;
 
\item[\tiny{$\blacksquare$}] with respect to another’s work: all text, diagrams, code, or ideas, whether verbatim, paraphrased or otherwise modified or adapted, have been duly attributed to the source in a scholarly manner, whether from books, papers, lecture notes or any other student’s work, whether published or unpublished, electronically or in print.
\\
\end{itemize}
 
 
Signed:\\
\rule[1em]{25em}{0.5pt}  % This prints a line for the signature
 
Date:\\
\rule[1em]{25em}{0.5pt}  % This prints a line to write the date
}
\clearpage  % Declaration ended, now start a new page

%% ----------------------------------------------------------------
% The "Funny Quote Page"
\pagestyle{empty}  % No headers or footers for the following pages

\null\vfill
% Now comes the "Funny Quote", written in italics
\textit{``We feel free because we lack the very language to articulate our unfreedom.''}

\begin{flushright}
Slavoj \v{Z}i\v{z}ek
\end{flushright}

\vfill\vfill\vfill\vfill\vfill\vfill\null
\clearpage  % Funny Quote page ended, start a new page
%% ----------------------------------------------------------------

\setstretch{1.3}  % Reset the line-spacing to 1.3 for body text (if it has changed)

% The Acknowledgements page, for thanking everyone
\acknowledgements{
\addtocontents{toc}{\vspace{1em}}  % Add a gap in the Contents, for aesthetics

Dedicated to my mother, who instilled in me the value of education.

With thanks to Marc van Dongen for his support throughout the research process.

}
\clearpage  % End of the Acknowledgements
%% ----------------------------------------------------------------

\pagestyle{fancy}  %The page style headers have been "empty" all this time, now use the "fancy" headers as defined before to bring them back


%% ----------------------------------------------------------------
\lhead{\emph{Contents}}  % Set the left side page header to "Contents"
\tableofcontents  % Write out the Table of Contents
\addtocontents{toc}{\vspace{2em}}  % Add a gap in the Contents, for aesthetics

\clearpage
%% ----------------------------------------------------------------
\mainmatter	  % Begin normal, numeric (1,2,3...) page numbering
\pagestyle{fancy}  % Return the page headers back to the "fancy" style

% Include the chapters of the thesis, as separate files
% Just uncomment the lines as you write the chapters

% !TeX spellcheck = en_GB
\chapter{Introduction}
\lhead{\emph{Introduction}}

This report and implementation of Locality Sensitive Bloom Filter contains several distinct parts:

\begin{itemize}
	\item A recursion based LSBF in Python
	\item A comparison between LSBF and other search methods
	\item Research that was done into implementing the project on an FPGA
	\item A critical analysis of this implementation of LSBF and improvements that can be made in future research
\end{itemize}

The aim of this report is to prove that an efficient implementation of LSBF is possible and to point to the research in the field of hashing algorithms that inspired this investigation. This report also contains recommended reading for future improvements that can be made to the provided implementation of LSBF.
 % Introduction

% !TeX spellcheck = en_GB
\chapter{Analysis}
\section{Project Objectives}
This project started as an investigation into hashing, Haskell and how C$\lambda$ash can be used to program an FPGA with Haskell code. During the research phase of this project, it was noted that the student would be required to have knowledge of Hardware Definition Language (HDL) to effectively understand how to program an FPGA. Therefore the specification of this project was adapted.

The objectives of this project are to:
\begin{itemize}
	\item demonstrate how one would set-up the provided FPGA for programming
	\item provide an implementation of a hashing algorithm that would be suited to the FPGA hardware
\end{itemize}

\section{Setting up the FPGA }
The provided FPGA was the DE0 Nano\cite{DE0NanoWebpage}, a compact FPGA suited to education. Resources exist on the manufacturer webpage as well as the provided software on how to set up the device. In the interest of making it  % Background Theory 

% !TeX spellcheck = en_GB
\chapter{Design}
\lhead{\emph{Design}}

\section{Algorithm Design}
As the project was originally intended for Haskell, it was decided that a recursive implementation of LSBF would be written for ease of porting in the future.

Nested bit arrays were used as a space efficient way to store the set bits. From a high-level perspective, bits are set based on the output of the hash of the input. The number of hashes is important in determining run-time and likelihood of false positive. If a minimum false positive rate is desired while optimising the number of hashes, then $k$ depends on $m$ and the expected $n$ as given below.\cite{ProbabilityAndComputing}
\begin{center}
	$ k = \dfrac{m}{n}\ln 2 $
\end{center}
We can calculate the optimum number of hashes to use based on our tolerated probability of a false positive rate $p$ using the following formulae:\cite{1195150}

\begin{center}
	$p=\left( 1-e^{-(\frac{m}{n}\ln 2)\frac{n}{m}} \right)^{\frac{m}{n}\ln 2} $
\end{center}

which simplifies to

\begin{center}
$\ln p = -\frac{m}{n}(\ln 2)^2$
\end{center}

meaning that

\begin{center}
$m = -\dfrac{n\ln p}{(\ln 2)^2}$
\end{center}

Finally, $k$ for our desired $p$ can be derived as

\begin{center}
	$k = -\dfrac{\ln p}{\ln 2}$
\end{center}

Here, we can see that for a given $p$, $k$ only depends on $p$ and $m \propto n$.

For ease of implementation two common hashes were used to prove how LSBF would work, however suggestions on more suitable hashes will follow in chapter \ref{Evaluation}: Evaluation. Additionally the above formulae work for regular Bloom filters but would need to be adapted for LSBF.

\section{Pseudo code}

While Mitzenmacher's Distance-Sensitive Bloom Filter has both a false positive and false negative rate, this version of LSBF makes an effort to eliminate false negative by use of rounding. In the pseudo code notation, we will use $\alpha$ to denote the upper bound for locality and $\beta$ for the rounding distance.

\begin{algorithm}[H]
	\KwIn{$ k, m , \alpha, \beta $}
	\KwResult{ An LSBF Object}
	\If{$k > 0$ and $m > 0$}{
		initialize an array of $m$ zeros\;
		store $\alpha, \beta, k$ in the object\;
	}
	\caption{LSBF Set Up}
\end{algorithm}

Adding an element to the LSBF is trivial in abstract. We assume that a function ``round'' exists that rounds to the nearest $\beta$, the second parameter:

\begin{algorithm}[H]
	\KwIn{input}
	\KwResult{None}
	\ForEach{hash, up to $k$}{
		Set a bit in the object's array to 1 based on the result of $hash(round(input, \beta))\mod{m}$\;
	}
\caption{Adding an element to the LSBF}
\end{algorithm}

In order to make the next section of code easier to understand, we will state a function that hashes for each $k$ and returns True when all checked bits are set to one.

Checking if a value is in the LSBF is, in this implementation, a recursive check of the space from $input - \alpha$ to $input + \alpha$ with a reduction of $\alpha$ by $\beta$ in each successive recursion until $\alpha$ is less than $\beta$.

\begin{algorithm}
	\KwIn{input, $\alpha$}
	\KwResult{True when an iteration shows all checked bits set to one, False otherwise}
	\eIf{$\alpha < \beta$}{
		$i$ := 0\;
		\While{$i < k$}{
			\eIf{the bit given by $hash_i$(input) is one}{
				$i$ := $i$ + 1\;
			}{
				\KwRet{False}
			}
		}
		\KwRet{True}
	}{
		$i$ := 0\;
		\While{$i < k$}{
			\eIf{the bit given by $hash_i$(input-$\beta$) is one}{
				$i$ := $i$ + 1\;
			}{
				
			}
		}
	}
\end{algorithm}
 % Experimental Setup

% !TeX spellcheck = en_GB
\chapter{Implementation}
\lhead{\emph{Implementation}}

\section{Language and Libraries}
The programming language Python was chosen for its ease of prototyping and vast libraries covering most required functionality. Python version 2.7 was used as most libraries support it and there is extensive documentation for it online.

The libraries $math$ and $hashlib$ are native to Python and imported for some basic mathematical functionality and to make use of some built-in hashing algorithms. The library ``$bitarray$'' is an efficient object type representing an array of booleans. It is implemented in C and abstracted to Python for use in Python programs. \cite{bitarray} It offers much functionality but the small memory footprint is of most use to this project.

Regarding possible future work a library, MyHDL, is available for Python. The most recent release adds support for Python 3. This library allows for code written in Python to be converted to VHDL for writing to an FPGA or ASIC. \cite{myhdl} As previously mentioned, to program with VHDL in mind would require experience in VHDL. Nonetheless, this package demonstrates that it is possible to effectively convert a Python MyHDL design to VHDL. MyHDL could become an integrated part of an effective workflow for a student that wishes to investigate this further.

\section{Development and Testing}
A class was written in Python that implements the algorithm from Chapter 3 and is attached at Appendix \ref{appendix:A}. For simplicity and to avoid clashes 65,536 bitarrays of length 65,536 are used for the Bloom filter, giving $m = 4,294,967,296$. The bit array in this case uses around 500 Megabytes. This is easy to use as we can just take the first 16 bits of a hash to find the position of the first array and the next 16 bits to select a bit in that array. In retrospect, using modulo on the result of the hash would have been an easier solution.
 
Two hash functions, SHA1 and MD5 were used. Chapter \ref{Evaluation} contains the criticisms of this approach and suggested alternative.
 % Experiment 1

% !TeX spellcheck = en_GB
\chapter{Evaluation}
\lhead{\emph{Evaluation}}
\label{Evaluation}

\section{Hash Functions}

Two hash functions were chosen to generate the hashes, SHA1 and MD5. MD5 is not considered cryptographically safe, but this doesn't matter in a Bloom filter, we just need a hash function that is uniform in its distribution and independent. MD5 is also a quick cryptographic hash so is of particular use to us in a Bloom filter. That being said, there is strong evidence that MurmurHash is a good hash to use in Bloom filters \cite{MurmurvsCryptoSpeed} despite not being considered a cryptographic hash

Looking at real world implementations of Bloom filters, it's clear that a better approach would've been to use the whole length of a 64-bit hash, splitting it up into four 16 bit indices.\cite{SquidBloom}

Considering all this, it is extremely likely that better results for LSBF would be obtained by making use of the aforementioned techniques.

\section{Speed Comparison}

A linear search has a $\mathcal{O}(n)$ time complexity on average which is comparable to the non-LSBF search function that was implemented for comparison, attached at Appendix \ref{appendix:C}.

LSBF on the other hand runs in $\mathcal{O}(\dfrac{k\alpha}{\beta}+k); 0 \leq \alpha < \beta; k \geq 1$ worst case with a constant space complexity of $\mathcal{O}(m)$. However, if the hash algorithm is computationally expensive (like SHA1) each $k$ will take a long time.

Using the timing demonstration in Appendix \ref{appendix:C}, we can see that for a large $n$ LSBF would start to outperform linear search. However, it should be noted that this approach to testing execution time is influenced by system load and is a naive implementation. The tests were re-run using Python's $timeit$ which only takes into account CPU time and is best practice for execution time comparison. The following results were obtained:

\begin{tabular}{|c|c|}
	\hline 
	\textbf{Algorithm} & \textbf{Execution Time} \\ 
	\hline 
	Iterative & 42.044 $ \mu $s \\ 
	\hline 
	LSBF (2 Hashes)& 63.502 $ \mu $s \\ 
	\hline 
	LSBF (MD5 only)& 53.435 $ \mu $s \\ 
	\hline 
\end{tabular} 

Each method was run using $timeit.repeat$ with a value $number=10000$ and randomly generated numbers in the range 0 -- 10000 tested as the query and 1000 values stored. It appears here that Iterative performs the best but when we increase $ n $ to 100,000 we can see something interesting.

\begin{tabular}{|c|c|}
	\hline 
	\textbf{Algorithm} & \textbf{Execution Time} \\ 
	\hline 
	Iterative & 58.434 $ \mu $s \\ 
	\hline 
	LSBF (2 Hashes)& 15.988 $ \mu $s \\ 
	\hline 
	LSBF (MD5 only)& 9.821 $ \mu $s \\ 
	\hline 
\end{tabular} 

Here, LSBF vastly outperforms an iterative approach, likely due to the fact that $n$ has increased beyond the range of the numbers we're picking from. This shows how effective LSBF is when a positive result is likely. With the range of numbers randomly selected increased to 0 -- 10,000,000 we see this:

\begin{tabular}{|c|c|}
	\hline 
	\textbf{Algorithm} & \textbf{Execution Time} \\ 
	\hline 
	Iterative & 6941.92886 $ \mu $s \\ 
	\hline 
	LSBF (2 Hashes)& 29.389 $ \mu $s \\ 
	\hline 
	LSBF (MD5 only)& 19.419 $ \mu $s \\ 
	\hline 
\end{tabular} 

This final result shows that for a large range and large $n$, LSBF still outperforms an iterative search. It should be noted, however, that false positives weren't checked in these timings. % Experiment 2

% !TeX spellcheck = en_UK
\chapter{Conclusions}
 % Results and Discussion

%\input{Chapters/Chapter7} % Conclusion

%% ----------------------------------------------------------------
% Now begin the Appendices, including them as separate files

\addtocontents{toc}{\vspace{2em}} % Add a gap in the Contents, for aesthetics

\appendix % Cue to tell LaTeX that the following 'chapters' are Appendices

% !TeX spellcheck = en_GB
\chapter{LSBF.py}
\lhead{\emph{Appendix A: LSBF.py}}
\label{appendix:A}

\lstset{language=Python, tabsize=3}
\begin{lstlisting}
import hashlib
import math

from bitarray import bitarray

class LocalitySensitiveBloomFilter:
	MAX_HASHES = 2 #The number of hashes currently implimented

	#Bloom Filter Basic Operations
	def __init__( self, floatPrecision, hashes,	bloomRange, resolution ):
		if not floatPrecision >= 0:
			return None # floatPrecision needs to be positive
		self.FLOAT_PRECISION = floatPrecision
		if not hashes > 0:
			return None
		if not hashes <= self.MAX_HASHES:
			return None
		self.NUM_HASHES = hashes
		self.LOCALITY_RANGE = bloomRange
		self.LOCALITY_RESOLUTION = resolution
		self.floatString = "{0:." + str(floatPrecision) + "f}"
		emptyArray = bitarray(2**16)
		emptyArray.setall(False)
		self.bloomArray = []

		for count in range( 0, 65536 ):
			self.bloomArray.append(emptyArray)

def setArrayBit( bloomArray, bitPosition ):
	bloomArray.bloomArray[bitPosition[0]][bitPosition[1]] = True
	return

def getArrayPos( bloomArray, input ):
	#This takes the first 16 + 16 bits of the hash and turns it into
	#a tuple, the position of a bit
	return ( int(bin(int(input, 16))[2:].zfill(8)[0:16], 2),
		int(bin(int(input, 16))[2:].zfill(8)[16:32], 2) )

def addToBloom( bloomArray, input ):
	input = bloomArray.floatString.format(input)
	bloomArray.setArrayBit(bloomArray.getMD5HashPosition(input))
	if bloomArray.NUM_HASHES > 1:
		bloomArray.setArrayBit(bloomArray.getSHA1HashPosition(input))

def checkInBloom( bloomArray, input ):
	input = bloomArray.floatString.format(input)
	if bloomArray.getMD5HashPresence( input ):
		if bloomArray.NUM_HASHES > 1:
			return bloomArray.getSHA1HashPresence( input )
		else:
			return True
	return False

def localityBloomCheck( bloomArray, input, range=0, recursionDepth=0 ):
	if (range == 0) and not (recursionDepth == bloomArray.LOCALITY_RANGE / 
		bloomArray.LOCALITY_RESOLUTION):
		range = bloomArray.LOCALITY_RANGE
	if (range < bloomArray.LOCALITY_RESOLUTION):
		return bloomArray.checkInBloom( input )
	else:
		if not bloomArray.checkInBloom( input - range ):
			if not bloomArray.checkInBloom( input + range ):
				if not bloomArray.localityBloomCheck( input,
					range - bloomArray.LOCALITY_RESOLUTION,
					recursionDepth+1 ):
					return False
	return True

#Hash Functions
#MD5 Based
def getMD5HashPosition( bloomArray, input ):
	m5 = hashlib.md5()
	m5.update(str(input).encode('utf-8'))
	return bloomArray.getArrayPos(m5.hexdigest())

def getMD5HashPresence( bloomArray, input ):
	inputPosition = bloomArray.getMD5HashPosition( input )
	return bloomArray.bloomArray[inputPosition[0]][inputPosition[1]]

#SHA1 Based
def getSHA1HashPosition( bloomArray, input ):
	sha1 = hashlib.sha1()
	sha1.update(str(input).encode('utf-8'))
	return bloomArray.getArrayPos(sha1.hexdigest())

def getSHA1HashPresence( bloomArray, input ):
	inputPosition = bloomArray.getSHA1HashPosition( input )
	return bloomArray.bloomArray[inputPosition[0]][inputPosition[1]]
\end{lstlisting}	% Appendix Title

% !TeX spellcheck = en_GB
\chapter{A Guide to Set-up the DE0-Nano}

\section{Software}
The required software can be downloaded from the Altera website. Quarus Prime Lite is recommended, as well as ModelSim and the device drivers (Cyclone IV for the DE0-Nano) \cite{QuartusDownloadPage}

\section{USB-Blaster Installation}
The USB-Blaster is required to enable communication between the system and the FPGA device. On a Windows computer, this can be done by following these steps:

\begin{enumerate}
	\item Connect the DE0-Nano to the PC
	\item Open the Device Manager
	\item Open the properties page for the USB-Blaster and click "Update Driver..."
	\item Select “Browse my computer for driver software” and navigate to the installation folder of Quartus (we\textsc{\char13}ll call it \%QuartusDir\%). Find the folder \texttt{“\%QuartusDir\%\textbackslash\{version\}\textbackslash quartus\textbackslash drivers\textbackslash usb-blaster\textbackslash x32”} and select it.
	\item Click next and install the driver	
\end{enumerate}

The USB-Blaster will now be available in Quartus Prime, allowing the device's software to be changed.

\section{Example Code}
When the software is installed, sample code can be used to test that the system is set up correctly. The examples can be found in \\ \texttt{\%QuartusDir\%\textbackslash\{version\}\textbackslash quartus\textbackslash qdesigns}. Note that not all of these will be compatible with the DE0-Nano.

When compiling, the target board needs to be set. For the DE0-Nano, you may open the project settings (Ctrl+Shift+E) and click the \textquotedblleft Device/Board\dots\textquotedblright  option on the top-right corner. In the Device family, select Cyclone IV E and type ``E22F17C6'' in the Name Filter to filter to the DE0-Nano’s on-board FPGA. Accept the settings and now you can compile for the DE0-Nano.

To compile, click ``Compile'' (Ctrl+L). This may take some time to complete, depending on the complexity of the project being compiled.
When finished compiling, one may write to the FPGA using the Programmer (Tools \textgreater  Programmer). When in the programmer, open ``Hardware Setup\dots'' and select the \textquotedblleft USB-Blaster\textquotedblright. The FPGA software will be re-written by clicking \textquotedblleft Start\textquotedblright. Note that this will only program the FPGA and not the configuration device, so when the device is powered down the program will be wiped from the FPGA’s memory and the program on the configuration device will be re-written onto the device when the device is rebooted. This is useful, however, during testing as writing to the FPGA is a lot quicker than writing to the configuration device.

One may think of the configuration device as a bootstrapper, writing the program to the FPGA's main memory on boot.

\section{Integrating ModelSim-Altera}



 % Appendix Title

%\input{Appendices/AppendixC} % Appendix Title

\addtocontents{toc}{\vspace{2em}}  % Add a gap in the Contents, for aesthetics
\backmatter

%% ----------------------------------------------------------------
\label{Bibliography}
\lhead{\emph{Bibliography}}  % Change the left side page header to "Bibliography"
\bibliographystyle{unsrtnat}  % Use the "unsrtnat" BibTeX style for formatting the Bibliography
\bibliography{Bibliography}  % The references (bibliography) information are stored in the file named "Bibliography.bib"

\end{document}  % The End
%% ----------------------------------------------------------------